\documentclass{article}
\usepackage[utf8]{inputenc}
\usepackage{fancyhdr}
\usepackage{lastpage}
\usepackage{amsmath}

\title{Repechage 2021}
\date{February 2021}

\usepackage{geometry}
\geometry{
 letterpaper,
 left=20mm,
 right=20mm,
 top=20mm,
 bottom=20mm
 }

\setlength{\parindent}{0pt}

\pagestyle{fancy}
\fancyhf{}

\chead{Question 2 \\}
\rfoot{{\thepage} of \pageref{LastPage}}

\begin{document}
We first find $x+y+z$:
\begin{align*}
    (x + y + z)^2 &= x^2 + y^2 + z^2 + 2xy + 2yz + 2zx \\
    &= 24 + 2(-4) \\
    &= 16 \\
    x + y + z &= \pm 4.
\end{align*}
Note that:
$$(xy + yz + zx)(x + y + z) = 3xyz + xy^2 + yz^2 + zx^2 + x^2y + y^2z + z^2x = -4(x + y + z).$$
We use this in the cube of the sum:
\begin{align*}
    (x + y + z)^3 &= x^3 + y^3 + z^3 + 6xyz + 3(xy^2 + yz^2 + zx^2 + x^2y + y^2z + z^2x) \\
    &= (x^3 + y^3 + z^3 + 3xyz) + 3(3xyz + xy^2 + yz^2 + zx^2 + x^2y + y^2z + z^2x) - 6xyz \\
    &= 16 + 3(-4)(x + y + z) - 6xyz
\end{align*}
\begin{align}
(x + y + z)^3 = 16 - 12(x + y + z) - 6xyz
\end{align}
Consider the case when $x + y + z = -4$, substitute this into $(1)$, we obtain:
\begin{align*}
    (-4)^3 &= 16 - 12(-4) - 6xyz \\
    6xyz &= 128.
\end{align*}
Since $x,y,z$ are integers and $128$ is not divisible by $6$, this is a contradiction. Hence, $x+y+z\neq -4$. \\

Now, we know $x + y + z = 4$, substitute this into $(1)$, we obtain:
\begin{align*}
    (4)^3 &= 16 + 3(-4)(4) - 6xyz \\
    6xyz &= -96 \\
    xyz &= -16.
\end{align*}
Now, without loss of generality assume $x\ge y\ge z$ as the equations are symmetric. \\
Since $xyz = -16$, $x + y + z = 4$, and $x\ge y\ge z$, we know $z < 0$ and $x,y > 0$. \\
We also know that $x,y,z\mid 16$, so $x,y,z \in \{\pm1,\pm2,\pm4,\pm8,\pm16\}$. \\
We test values of $x$: \\
$x=16$ gives $(16,1,-1)$, the sum is too large. \\
$x=8$ gives $(8,2,-1)$ and $(8,1,-2)$, the sums are too large. \\
$x=4$ gives $(4,4,-1)$, $(4,2,-2)$, and $(4,1,-4)$, only $(4,2,-2)$ satisfies the sum. \\
$x=2$ gives $(2,2,-4)$ and $(2,1,-8)$, the sums are too small. \\
$x=1$ gives $(1,1,-16)$, the sum is too small. \\

Thus, when $x\ge y\ge z$, $(4,2,-2)$ is the only possible solution.
However, without the assumption $x\ge y\ge z$, all permutations of the three numbers are solutions.
Thus, the solutions are:
$$(x,y,z) = (4,2,-2),(4,-2,2),(2,4,-2),(2,-2,4),(-2,4,2),(-2,2,4).$$
\end{document}