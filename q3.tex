\documentclass{article}
\usepackage[utf8]{inputenc}
\usepackage{fancyhdr}
\usepackage{lastpage}
\usepackage{amsmath}
\usepackage{graphicx}

\title{Repechage 2021}
\date{February 2021}

\setlength{\parindent}{0pt}

\pagestyle{fancy}
\fancyhf{}

\chead{Question 3 \\}
\rfoot{{\thepage} of \pageref{LastPage}}

\begin{document}
\begin{figure}[htp]
    \centering
    \includegraphics[scale=1]{Q3_img}
    \label{fig:diagram}
\end{figure}
Join $AD$ and $CE$.
Let $Q$ be the intersection of $\overline{AD}$ and $\overline{CE}$. \\
\textbf{Now, prove $P$ is $Q$.} \\

Note that $ABCDE$ is a regular pentagon, so its interior angles all measure $108^\circ$. \\
Note that since $AE = ED$, $\triangle AED$ is isosceles with base $\overline{AD}$. Thus:
$$\angle ADE = \angle DAE = \frac{180^\circ-\angle AED}{2} = \frac{180^\circ-108^\circ}{2} = 36^\circ.$$
Similarly, since $ED = CD$ and $\angle CDE = 108^\circ$, $\angle CED = 36^\circ$. \\
Thus, $\angle ADE = \angle CED = 36^\circ$. \\
Since $Q$ is on $\overline{AD}$ and $\overline{CE}$, thus:
$$\angle QDE = \angle ADE = \angle CED = \angle QED = 36^\circ.$$

\iffalse
Since $\angle QDE = \angle ADE = \angle CED = \angle QED$, $\triangle EPD$ is isosceles with base $\overline{ED}$. Thus:
$$\angle EQD = 180^\circ - \angle QDE - \angle QED = 180^\circ - 36^\circ - 36^\circ = 108^\circ.$$
\fi
$Q$ is on $\overline{AD}$, use the exterior angle sum property:
$$\angle AQE = \angle QED + \angle QDE = 36^\circ + 36^\circ = 72^\circ.$$
Also, $\angle AEQ = \angle AED - \angle QED = 108^\circ - 36^\circ = 72^\circ$. \\
Thus, $\angle AEQ = \angle AQE = 72^\circ$, and thus $\triangle EAQ$ is isosceles with base $\overline{EQ}$.
Thus, $AQ = AE = AB$. \\
Similarly, due to symmetry, $CQ = CD = CB$. \\
Since $AQ = AB$, $Q$ is on $C_1$; $CQ = CB$, $Q$ is on $C_2$. \\
Thus, $Q$ is an intersection of $C_1$ and $C_2$. \\
$Q$ is not on the same side of $\overline{AC}$ as $B$, so $Q$ cannot be $B$. \\
\textbf{Thus, $Q$ must be $P$.} \\ \\ \\

\rightline{(Continued on next page.)}
\newpage
Join $PY$, $PX$, $CX$. Let the circle centred at $P$ passing through $E,D,X,Y$ be $\omega$.\\

Note that $\angle CPD = \angle APE = 72^\circ$. \\
$CD = CX$ as they are radii of $C_2$, and $PX = PD$ as they are radii of $\omega$. \\
Thus, $\triangle CPX \cong \triangle CDP$ by SSS congruence. \\
Thus, $\angle CPX = \angle CPD = 72^\circ$ \\
Since $P$ is on $\overline{CE}$:
$$\angle APX = 180^\circ - \angle APE - \angle CPX = 180^\circ - 72^\circ - 72^\circ = 36^\circ.$$
$PY = PE$ as they are radii of $\omega$, thus:
$$\angle PYE = \angle PEY = \angle AEP = 72^\circ.$$
Thus, $\angle EPY = 180^\circ - \angle PYE - \angle PEY = 36^\circ$,
$$\angle APY = \angle APE - \angle EPY = 72^\circ - 36^\circ = 36^\circ.$$
Since, $\angle APX = \angle APY = 36^\circ$, $PX = PY$ as they are radii of $\omega$, and $\triangle APX$ and $\triangle APY$ share segment $\overline{AP}$. $\triangle APX \cong \triangle APY$ by SAS congruence. \\
Thus, $AX = AY$.

\end{document}