\documentclass{article}
\usepackage[utf8]{inputenc}
\usepackage{fancyhdr}
\usepackage{lastpage}
\usepackage{amsmath}

\title{Repechage 2021}
\date{February 2021}

\setlength{\parindent}{0pt}

\usepackage{geometry}
\geometry{
 letterpaper,
 left=20mm,
 right=20mm,
 top=20mm,
 bottom=20mm
 }


\pagestyle{fancy}
\fancyhf{}

\chead{Question  8\\}
\rfoot{{\thepage} of \pageref{LastPage}}

\begin{document}
Starting with the King, moving in the clockwise direction, label the people from 1 to 2021. The original problem is converted to: find the number of circular permutations of the set $\{1,2, \dots , 2021\}$ without succession. (Succession is when a number is exactly 1 greater than the number before it, 1 after 2021 is a special succession.) \\

We will find an explicit formula for a circular permutation of the set $\{1,2,\dots , n\}$ without successions. Note that $1$ after $n$ is also a succession. \\

For a set $S$ define $|S|$ to be the cardinality of the set, that is, the number of elements of $S$. \\

Let $C$ be the set of circular permutations $(p_1,p_2,\dots , p_n)$ of the set $\{1,2,3,\dots , n\}$. \\
Let $C_i$ be the subset of $C$ where the pair $(p_i, p_{i + 1})$ is a succession, with $i = 1,2,3,\dots , n$. \\

Now, consider a set of $k$ indices $I = \{i_1,i_2, \dots , i_k \}$, with $1 \le i_1 < i_2 < \dots \le n$. \\
We now prove $\left| C_{i_1}\cap C_{i_2} \cap \dots \cap C_{i_k} \right|$ depends only on $k$ and is independent of the indices. \\

When the set $I$ is consists entirely of successive integers $l,l+1,\dots l + k - 1$  (note this is a circular permutation, so when the number exceeds $n$, take modulo $n$), the $k$ successions $(p_l,p_{l+1}), (p_{l+1},p_{l+2}), \dots, (p_{l+k - 1},p_{l+k})$ can be joined to form a block $b$ of $k + 1$ successive integers $(p_l,p_{l+1},p_{l+2}, \dots ,p_{l+k})$. Here, $\left| C_{i_1}\cap C_{i_2} \cap \dots \cap C_{i_k} \right|$ is just the number of circular permutations of the block $b$ and the remaining $n-k-1$ integers, which total to $n-k$ objects. Recall the number of circular permutations of $m$ objects is $\frac{m!}{m} = (m-1)!$. \\
Thus, $\left| C_{i_1}\cap C_{i_2} \cap \dots \cap C_{i_k} \right|$ =  $(n-k-1)!$ in this case. \\

When the set $I$ does not consist of successive integers, it can be partitioned into $d$ disjoint subsets $I_1,I_2,\dots , I_d$, with $d\in \{ 2,3,4,\dots,k \}$, such that each set contains the greatest possible number of successive integers. Thus, each of $I_1,I_2,\dots , I_d$, is either a set of successive integers or a set of one integer. Let $c_j = |I_j|$, with $\sum_{j=1}^d c_j = |I| = k$. \\

For any $j\in \{1,2,3,\dots , d \}$ the successions in subset $I_j$ can be joined to form a block $b_j$ of $c_j + 1$ successive integers, and an integer cannot be in two blocks, (otherwise, that contradicts the fact that each $I_j$ contains the greatest number of successive integers). Thus, the total number of integers in the blocks are:
$$\sum_{a=j}^d (c_j + 1) = d + \sum_{a=j}^d c_j = d + k.$$

Now, $\left| C_{i_1}\cap C_{i_2} \cap \dots \cap C_{i_k} \right|$ is just the number of circular permutations of the $d$ blocks and the remaining $n-k-d$ integers, which total to $n-k$ objects. There are $(n-k-1)!$ ways to permute the blocks and the remaining integers. \\

Thus, $\left| C_{i_1}\cap C_{i_2} \cap \dots \cap C_{i_k} \right|$ =  $(n-k-1)!$, which depends only on $k$. \\

Thus, the total number of circular permutations with at least $k$ successions is $(n-k-1)!$ multiplied by the number of ways to choose the indices $I$, which is $\binom{n}{k}$. This product equals $\binom{n}{k}(n-k-1)!$ for $k\in \{0,1,2,\dots, n-1 \}$. Note the special case when $k = n$, every number must be in succession, and the only possible circular permutation is $(1,2,3,\dots , n)$, and thus the total is just $1$. \\

By the inclusion-exclusion principle, the number of circular permutations of the set $\{1,2,3,\dots , n\}$ with exactly 0 successions is:
$$ \sum_{k=0}^{n-1} (-1)^k \binom{n}{k}(n-k-1)! + (-1)^n.$$
Thus, for the original problem, the number of reseatings is:
\begin{align*}
    &\sum_{k=0}^{2021-1} (-1)^k \binom{2021}{k}(2021-k-1)! + (-1)^{2021} \\
    = &\boxed{\sum_{k=0}^{2020} (-1)^k \binom{2021}{k}(2020-k)! + -1}.
\end{align*} 

\end{document}



is the set of successive integers $l,l+1,\dots l + c_j - 1$ (note this is a circular permutation, so when the number exceeds $n$, take modulo $n$), where $c_j = |I_j|$. The $c_j$ successions $(l,l+1), (l+1,l+2), \dots, (l+c_j - 1,l+c_j)$

By reseating the guests, the King is simply making a circular permutation of the set $\{ 1,2,3,\dots , 2021\}$, as the circular permutation can always be rotated to the King sits in his throne. \\

However, since everyone notices the person to their left is different from the one previously to their left, then the new circular permutation cannot contain a succession. \\

Consider the set $\{1,2,3,\dots , n\}$ permuted in a circle $(p_1,p_2,\dots , p_n)$, where its rotations are considered to be the same. Define such a permutation to be a \textbf{circular permutation}. \\
Note that in a circular permutation $(p_1,p_2,\dots , p_n)$, $p_{i + k} = p_{(i + k) \bmod n}$. \\

For a circular permutation $(p_1,p_2,\dots , p_n)$ of the set $\{1,2,\dots , n\}$, if there exists a pair $(p_i,p_{i+1})$ such that:
$$p_i + 1 \equiv p_{i + 1} \bmod n,$$

then define the pair $(p_i,p_{i+1})$ to be a \textbf{succession}. \\

Label the King 1, moving in the clockwise direction (to the king's left) label the guests from 2 to 2021. By reseating the guests, the King is simply making a circular permutation of the set $\{ 1,2,3,\dots , 2021\}$, as the circular permutation can always be rotated to the King sits in his throne. \\

The set $I$ can be partitioned into $d$ disjoint subsets $I_1,I_2,\dots , I_d$, each of which is either a set of successive integers or a set of one integer. 