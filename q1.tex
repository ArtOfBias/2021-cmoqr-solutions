\documentclass{article}
\usepackage[utf8]{inputenc}
\usepackage{fancyhdr}
\usepackage{lastpage}
\usepackage{amsmath}

\title{Repechage 2021}
\date{February 2021}

\setlength{\parindent}{0pt}

\pagestyle{fancy}
\fancyhf{}

\chead{Question 1 \\}
\rfoot{{\thepage} of \pageref{LastPage}}

\begin{document}
When $p(x) = 0$, $p(x+p(x)) = 0 = x^2\cdot 0 = x^2p(x)$. Thus $p(x) = 0$ is a solution. \\

Next, consider when $p(x)\neq 0.$ \\
For the equation to hold, the degree of $p(x + p(x))$ must equal that of $x^2p(x)$. \\
Assume the degree of $p(x)$ is 0, then the degree of $p(x + p(x))$ is 0, and the degree of $x^2p(x)$ is 2. Thus, the degree of $p(x)$ cannot be 0. \\
Assume the degree of $p(x)$ is 1, then the degree of $p(x + p(x))$ is 1, and the degree of $x^2p(x)$ is 3. Thus, the degree of $p(x)$ cannot be 1. \\
Assume the degree of $p(x)$ is 2, then the degree of $p(x + p(x))$ is 4, and the degree of $x^2p(x)$ is 4. In this case, the equation can be true. \\
Assume the degree of $p(x)$ is $n$ where $n\ge3$, then $p(x + p(x))$ is greater than or equal to $n^2$, and the degree of $x^2p(x)$ is $n + 2$. Since $n\ge3$, $n^2\neq n + 2. $ Thus, the degree of $p(x)$ cannot be greater than or equal to $3$. \\

Thus, the degree of $p(x)$ can only be $2$. Let $p(x) = ax^2 + bx + c$, with real numbers $a,b,c$ and $a\neq 0$. Substitute into the original equation, thus:
\begin{align*}
    a(x + ax^2 + bx + c)^2 + b(x + ax^2 + bx + c) + c &= x^2(ax^2 + bx + c) \\
    a(ax^2 + (b+1)x + c)^2 + b(ax^2 + (b+1)x + c) + c &= ax^4 + bx^3 + cx^2.
\end{align*}
Compare the coefficients of $x^4$ of both sides. This produces $a^3 = a$. Since $a\neq 0$, the solutions are $a=\pm 1$. \\
Compare the coefficients of $x^3$ on both sides, note that since $a=\pm 1$, $a^2=1$:
\begin{align*}
    2a^2(b+1) &= b \\
    2b + 2 &= b \\
    b &= -2.
\end{align*}
Now, consider the coefficients of $x^1$:
\begin{align*}
    2ac(b+1) + b(b+1) &= 0 \\
    2ac(-1) + (-2)(-1) &= 0 \\
    ac-1 &= 0 \\
    c = \frac{1}{a}.
\end{align*}
Since $a=\pm 1$, $c=1$ when $a=1$, and $c=-1$ when $a=-1$. \\

Thus, the possible polynomials $p(x)$ are:
\begin{align*}
    p(x) &= 0 \\
    p(x) &= x^2 - 2x + 1 \\
    p(x) &= -x^2 -2x - 1.
\end{align*}
\end{document}




Since $x=\pm 1$, $x^2=1$:
Compare the coefficients of $x^1$ on 


Next, assume $a = 1$, c
the right side has no constant term, $c=0$:

Compare the coefficients of $x^4$ of both sides. This produces $a^3 = a.$
Now, consider the 

\\
Denote by $\deg(f(x))$ the degree of the polynomial $f(x)$, note that this would output a non-negative integer. \\
First, consider the case when $\deg(p(x)) \le 1$. This gives $\deg(p(x) + x) = 1$ We thus have:
\begin{align*}
    \deg(p(x + p(x))) &= deg(p(x)) \\
    \deg(x^2 p(x)) &= 2 + \deg(p(x)) \ge 2
\end{align*}
Thus, the polynomials $p(x) + x$ and $x^2 p(x)$ have different degrees, and cannot be equal. No solutions are obtained in this case.
Now, we consider $\deg(p(x)) > 1$, this gives us:
\begin{align*}
    \deg(p(x) + x) &=  \max(\deg(p(x)), 1) = \deg(p(x)) \\
    \deg(x^2 p(x)) &= \deg(p(x)) + 2
\end{align*}
For the two polynomials to be equal, they must have equal degrees. Thus:
$$\deg(p(x)) = 2 + \deg(p(x))$$
Next, consider when $p(x)\neq 0.$ \\
Denote by $\deg(f(x))$ the degree of the polynomial $f(x)$
For the equation to hold, $\deg(p(x + p(x)))$ equal $\deg(x^2p(x))$. \\
Assume $\deg(p(x)) = 0$, then the degree of $p(x + p(x))$ is 0, and the degree of $x^2p(x)$ is 2. Thus, the degree of $p(x)$ cannot be 0. \\
Assume $\deg(p(x)) = 1$, then the degree of $p(x + p(x))$ is 1, and the degree of $x^2p(x)$ is 3. Thus, the degree of $p(x)$ cannot be 1. \\
Assume $\deg(p(x)) = 2$, then the degree of $p(x + p(x))$ is 4, and the degree of $x^2p(x)$ is 4. In this case, the equation can be true. \\
Assume the degree of $p(x)$ is $n$ where $n\ge3$, then $p(x + p(x))$ is greater than or equal to $n^2$, and the degree of $x^2p(x)$ is $n + 2$. Since $n\ge3$, $n^2\neq n + 2.$ \\