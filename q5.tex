\documentclass{article}
\usepackage[utf8]{inputenc}
\usepackage{fancyhdr}
\usepackage{lastpage}
\usepackage{amsmath}
\usepackage{amssymb}

\title{Repechage 2021}
\date{February 2021}

\usepackage{geometry}
\geometry{
 letterpaper,
 left=35mm,
 right=35mm,
 top=30mm,
 bottom=30mm
 }

\setlength{\parindent}{0pt}

\pagestyle{fancy}
\fancyhf{}

\chead{Question 5 \\}
\rfoot{{\thepage} of \pageref{LastPage}}

\begin{document}
% Let A denote Alphonse, B denote Beryl. \\
Let $E_2 \subset \mathbb{N}$ denote the set of positive integers where the exponent of $2$ in its prime factorisation is even. That is:
$$\forall k \in E_2 \implies \exists m,n \in \{ \mathbb{Z} \cap [0, \infty) \} : k = 2^{2m}(2n + 1).$$

Let $O_2 \subset \mathbb{N}$ denote the set of positive integers where the exponent of $2$ in its prime factorisation is odd. That is:
$$\forall k \in O_2 \implies \exists m,n \in \{ \mathbb{Z} \cap [0, \infty) \} : k = 2^{2m + 1}(2n + 1).$$
%For any integer in $E_2$, call it 2-even, and any integer in $O_2$ 2-odd. \\
Note that $E_2 \cap O_2 = \O$ and $E_2 \cup O_2 = \mathbb{N}$ as the exponent of 2 in a prime factorisation can only be even or odd. \\ \\

\textbf{Lemma 1:}
\setlength{\parindent}{15pt} \setlength{\parskip}{5pt}
\par If an integer $n \in E_2$, then $\frac{n}2, 2n \in O_2$. \setlength{\parskip}{0pt}
\par If an integer $n \in O_2$, then $\frac{n}2, 2n \in E_2$. \\

\setlength{\parindent}{0pt}
\textbf{Proof:} \\

When an integer is multiplied or divided by $2$, the exponent of $2$ in it prime factorisation increases or decreases by $1$, which changes its parity and hence which set the integer belongs to. \\ \\

\textbf{Lemma 2:}
\begin{align*}
    &\text{Given rectangles } (1\times n, 1\times n) \text{, where } n \in E_2 \text{. The current player cannot win.} \tag{i} \\
    &\text{Given rectangles } (1\times m, 1\times m) \text{, where } m \in O_2 \text{. The current player can win.} \tag{ii} \\
    &\text{Given rectangles } (1\times a, 1\times b) \text{, where } a,b \in \mathbb{N} \text{ and } a \neq b \text{. The current player can win.} \tag{iii}
\end{align*}


\textbf{Proof:} \\

First, consider $(1 \times 1, 1 \times 1)$, this is the only way the game can end, when a player receives these rectangles, they cannot make a cut and loses. Also, since $1 = 2^0$ this is obviously case (i).\\

\textbf{In a case (i)} that is not $(1 \times 1, 1 \times 1)$, the player must make a cut, suppose it is cut into the following:
$$ (1 \times n, 1 \times k, 1 \times l) \text{ where } k + l = n,\ k \le l < n.$$

The player must then discard a rectangle. If one of the latter two are discarded, this becomes a case (iii) for the other player as $k,l \neq n$. If $1 \times n$ is discarded, then it becomes case (iii) for the other player if $k\neq l$. Now, consider the case when $k = l$. Since $k + l = n$, this means that $k = l = \frac{n}{2}.$ Since $n \in E_2$, $\frac{n}2 \in O_2$ by Lemma 1. Thus, this is a case (ii). \\

\textbf{In case (ii)}, it can be proven that the current player can always give the next player a case (i). The current player takes one of the rectangles and cuts it into:
$$ \left( 1 \times m, 1 \times \frac{m}2, 1 \times \frac{m}2 \right).$$

Since $m \in O_2$, the exponent of 2 in its prime factorisation is at least $1$, so $\frac{m}2$ is an integer, and the above cut can be made. The player then discards $1 \times m$, giving the next player $\left( 1 \times \frac{m}2, 1 \times \frac{m}2 \right)$. Since $m \in O_2$, $\frac{m}2 \in E_2$ by Lemma 1. Thus the next player is given a case (i).

\newpage

\textbf{In case (iii)} the next player can also be given a case (i). First assume without loss of generality that $a < b$. Then, this has two cases:
\begin{align*}
    a &\in E_2 \text{ or } \tag{iii.i} \\
    a &\in O_2. \tag{iii.ii}
\end{align*}
In case (iii.i), since $b>a$, the player cuts the rectangles into:
$$ (1 \times a, 1 \times a, 1 \times (b-a)).$$
Then, $1 \times (b-a)$ is discarded to produce $(1 \times a, 1 \times a)$. Since $a \in E_2$, this is case (i). \\

In case (iii.ii),  $1\times a$ is cut in half and $1\times b$ is discarded to produce $(1 \times \frac{a}2, 1 \times \frac{a}2)$. \\
$\frac{a}2 \in E_2$ by Lemma 1, so this is also case (i). \\

Thus, in cases (ii) and (iii), it is always possible to perform an action that gives the next player a case (i), while any action on case (i) can only give the next player case (ii) or (iii). \\
Note that the total area of the two rectangles is monovariant as it decreases every time a cut is made and a rectangle discarded. Thus, the game cannot continue infinitely and must eventually reach the position $(1 \times 1, 1 \times 1)$, which has the least possible total area. \\

Thus, when given a case (ii) or (iii) position, the current player can force the next player into a case (i) position, which will return to them a case (ii) or a case (iii). Thus, the current player in case (ii) or (iii) cannot lose as $(1 \times 1, 1 \times 1)$ is a case (i), thus a player can always win given cases (ii) and (iii). \\

Similarly, when the current player has a case (i), the next player can only be given a case (ii) or (iii), where they can then force the current player into a case (i) position. This continues and eventually the current player is given $(1 \times 1, 1 \times 1)$ and loses. \\ \\


\textbf{Lemma 3:}

\setlength{\parskip}{5pt}
For any integer $e \in E_2$, when the current player is given rectangles $(1\times e, 2\times 2 e)$, the next player can always win. \setlength{\parskip}{0pt} \\

\setlength{\parindent}{0pt}
\textbf{Proof:} \\

Let A be the current player and B the next player. \\
First, consider when A cuts the larger rectangle perfectly in half and discards $1\times e$. This gives:
$$ (1\times 2e, 1\times 2e) \text{ or } (2\times e, 2\times e).$$

In the first case, this is a case (ii) position in Lemma 2, and so B can win. In the second case, B cuts $2\times e$ in half and  discards the remaining $2\times e$ to give A $(1\times e, 1\times e)$, by Lemma 1 and Lemma 2 case (i), A loses and B wins. \\

If A cuts $1\times e$ and discards $2\times 2e$, to produce $\left(1\times \dfrac{e}2, 1\times \dfrac{e}2 \right)$, by Lemma 1 and case (ii) of Lemma 2, B can win. \\

If A cuts the larger rectangle in half and discards one half to produce $(1\times e, 1\times 2e)$, then by case (iii) of Lemma 2, B can win. \\

\newpage
If A cuts $1\times e$ into unequal pieces and discards the larger rectangle, then B is given:
$$(1\times m, 1\times n)\text{ with } m\neq n, m + n = e.$$

By case (iii) of Lemma 2, B can win. \\

If A cuts $1\times e$ into two pieces and discards one of them, then B is given:
\begin{equation*}
    (1\times m, 2\times 2e)\text{ with }  m < e.
\end{equation*}
This has two cases: \\
\textbf{If} $m \in O_2$, B cuts $1\times m$ in half and discards the $2\times 2e$ rectangle, giving A the following:
$$\left( 1\times \frac{m}2, 1\times \frac{m}2 \right).$$

By Lemma 1 and case (i) of Lemma 2, A loses and B wins. \\

\textbf{If} $m \in E_2$, call this case (1). B cuts the $2\times 2e$ rectangle into $(2\times 2m, 2\times 2(e-m))$ and discards the latter, this is possible as $m<e$. Then A is given:
$$(1\times m, 2\times 2 m) \text{ with } m \in E_2.$$

This is a smaller version of the original rectangles in this lemma. \\

Suppose instead that A cuts $2\times 2e$ into two pieces and discards one of them, then B is given:
\begin{equation*}
    (1\times e, 2\times m)\text{ with }  m < 2e.
\end{equation*}
This has two cases: \\
\textbf{If} $m \in E_2$, B cuts $2\times m$ in half parallel to the side with length $m$ and discards $1\times e$, giving A the following:
$$\left( 1\times m, 1\times m \right).$$

By case (i) of Lemma 2, A loses and B wins. \\

\textbf{If} $m \in O_2$, call this case (2). B cuts the $1\times e$ rectangle into $\left( 1\times \dfrac{m}2, 1\times \left( e - \dfrac{m}2 \right) \right )$ and discards the latter, this is possible as $m<2e$. Then A is given:
$$\left(1\times \dfrac{m}2, 2\times m\right) \text{ with } \frac{m}2 \in E_2.$$

Since $\dfrac{m}2 < e$, this is a smaller version of the original rectangles in this lemma. \\

Now, consider cases (1) and (2). If A decides only performs these two moves, leading to case (1) or (2), then B can force them into the Lemma 3 position. Since the total area is strictly decreasing, A will eventually be given $(1\times 1, 2\times 2)$, which is a Lemma 3 position, where they must give B $(1\times 1, 1\times 2)$ or $(1\times 2, 1\times 2)$, by cases (ii) and (iii) of Lemma 2, B wins. \\

Finally, if A discards $1\times e$ and cuts $2\times 2e$, giving B:
$$(2\times m, 2\times n)\text{ with } m + n = 2e.$$
If at least one of $m$ and $n$ are in $E_2$, then B simply cuts that rectangle into $( 1\times m, 1\times m )$ or $( 1\times n, 1\times n )$, depending on which of $m,n$ is in $E_2$. By case (i) of Lemma 2, A loses and B wins. \\

Therefore, $m,n \in O_2$. Since $2e \in O_2$, without loss of generality assume $m > n$. However, similar to Lemma 2, B can then cut and discard on $(2\times m, 2\times n)$ to give A $(2\times n, 2\times n)$. To avoid giving B a rectangle with a side in $E_2$, A must cut then create another two unequal rectangles $(2\times a, 2\times b)$, with $a,b \in O_2,\ a < b$. B continues to give A two equal rectangles with side lengths in $O_2$, this is possible as long as $a,b \in O_2,\ a < b$. \\

Since the area is strictly decreasing, A will eventually be given $(2\times 2, 2\times 2)$, at which point A must give B $(1\times 2, 2\times 2)$ or $(1\times 2, 1\times 2)$. Now, B cuts a $1\times 2$ rectangle to give A $(1\times 1, 1\times 1)$, and A loses. \\

\textbf{Thus, B can always win when A is given $(1\times e, 2\times 2 e)$.}
\section*{Part (a)}
\boxed{\text{Beryl wins.}} \\

The prime factorisation of $2020$ is $2^2\cdot 5 \cdot 101$, so $2020 \in E_2$. Note that $4040 = 2\times 2020.$\\
By Lemma 3, no matter how Alphonse cuts and discards the rectangles, Beryl can always win.

\section*{Part (B)}
\boxed{\text{Alphonse wins.}} \\

Assume $(100\times 100, 100\times 500)$ is a losing position, that is, any position given to Beryl must be a winning position. \\
Based on this assumption, $(100\times 100, 100\times 100)$ is a winning position. The only possible moves for Beryl are the following: \\
(1) Cut $100\times 100$ into $(a\times 100, b\times 100)$, with $a+b=100$ and $a,b<100$, and discard one of them to give Alphonse the position:
$$(m\times100,100\times100),\text{ with } m < 100.$$

(2) Cut $100\times 100$ into $(a\times 100, b\times 100)$, with $a+b=100$ and $a,b<100$, and discard $100\times100$ to give Alphonse the position:
$$(a\times100,b\times100)\text{ with } a + b = 100\text{ and } a,b<100.$$

At least one of these is a losing position, else Beryl cannot win from $(100\times 100, 100\times 100)$. \\
However, both of these positions can be obtained from the original: \\
since $m<100$ cut the $100\times 500$ rectangle into $(m\times100,(500-m)\times 100)$ and discard the latter:
$$(m\times100,100\times100),$$
cut the $100\times100$ rectangle into $(a\times 100, b\times 100)$, with $a+b=100$ and $a,b<100$, and discard $100\times500$:
$$(a\times100,b\times100)\text{ with } a + b = 100\text{ and } a,b<100.$$
Since both of these positions can be obtained from $(100\times 100, 100\times 500)$, the assumption that it is a losing position is incorrect. Thus, $(100\times 100, 100\times 500)$ is a winning position. \\

\textbf{Thus, Alphonse wins from the position $(100\times 100, 100\times 500)$.}

\end{document}
Alphonse first cuts $100\times500$ into $(100\times 100, 100\times 400)$ and discards $100\times400$, giving Beryl $(100\times 100, 100\times 100)$. Suppose that this is a losing position, that is, Beryl has no move that can result in a win, then Alphonse simply wins with this move. \\

However, a

However, suppose Beryl can make a winning move from $(100\times 100, 100\times 100)$ and force Alphonse into a losing position. Beryl's can only cut one of the $100\times 100$ into $(a\times 100, b\times 100)$, with $a+b=100$ and $a,b<100$. Beryl can either discard one of them, to give Alphonse the position:

$$(m\times100,100\times100),$$
with $m$ some positive integer less than $100$, or Beryl can discard $100\times100$, giving Alphonse:
$$(a\times100,b\times100)\text{ with } a + b = 100\text{ and } a,b<100.$$
If Beryl can win, then at least one of those positions is a losing position. \\
However, if these are losing positions, Alphonse can give Beryl these positions. Since $m<100$, Alphonse can cut the $100\times 500$ rectangle into $(m\times100,(500-m)\times 100)$ and discard the latter, giving Beryl:
$$(m\times100,100\times100).$$
Alphonse can also cut the $100\times100$ rectangle into $(a\times 100, b\times 100)$, with $a+b=100$ and $a,b<100$, and discard $100\times500$. This gives Beryl:
$$(a\times100,b\times100)\text{ with } a + b = 100\text{ and } a,b<100.$$
Thus, any move made from $(100\times 100, 100\times 100)$ can also be made from $(100\times 100, 100\times 500)$, 
\textbf{Thus, Alphonse always has a winning strategy, }


Substitute a number $e \in E_2$ for 2020. The rectangles become. $(1\times e, 2\times 2e)$
First, consider when A cuts the larger rectangle perfectly in half and discards $1\times e$. This gives:
$$ (1\times 2e, 1\times 2e) \text{ or } (2\times e, 2\times e).$$

In the first case, this is a case (ii) position in Lemma 2, and so B can win. In the second case, B cuts $2\times e$ in half and  discards the remaining $2\times e$ to give A $(1\times e, 1\times e)$, by Lemma 1 and Lemma 2 case (i), A loses and B wins. \\

If A cuts $1\times e$ and discards $2\times 2e$, to produce $\left(1\times \dfrac{e}2, 1\times \dfrac{e}2 \right)$, by Lemma 1 and case (ii) of Lemma 2, B can win. \\

If A cuts the larger rectangle in half and discards one half to produce $(1\times e, 1\times 2e)$, then by case (iii) of Lemma 2, B can win. \\

If A cuts $1\times e$ into unequal pieces and discards the larger rectangle, then B is given:
$$(1\times m, 1\times n)\text{ with } m\neq n, m + n = e.$$

By case (iii) of Lemma 2, B can win.
\newpage
If A cuts $1\times e$ into unequal pieces and discards one of them, then B is given:
\begin{equation}
    (1\times m, 2\times 2e)\text{ with }  m < e .\tag{1}
\end{equation}
This has two cases. \\

If $m \in O_2$, B cuts $1\times m$ in half and discards the $2\times 2e$ rectangle, giving A the following:
$$\left( 1\times \frac{m}2, 1\times \frac{m}2 \right).$$
By Lemma 1 and case (i) of Lemma 2, A loses and B wins. \\

If $m \in O_2$, then 

\newpage
Let an unordered pair $(w \times x, y \times z)$ denote the rectangles that the current player is given. Note that the order of $w$ and $x$ and $y$ and $z$ does not matter. For example, the position with rectangles $(1\times 5, 3\times 4)$ is identical to the position with rectangles $(4\times 3, 1\times 5)$.

\newpage
\textbf{Lemma 2}: \\

Let $e_1,e_2,e_3,e_4$ be arbitrary (not necessarily distinct) 2-even integers. \\
Lot $o_1,o_2,o_3,o_4$ be arbitrary (not necessarily distinct) 2-odd integers. \\
All possible rectangles pairs are one of the following cases:
\begin{align}
    &(e_1 \times e_2 , e_3 \times e_4) \\
    &(e_1 \times e_2 , e_3 \times o_4) \\
    &(e_1 \times e_2 , o_3 \times o_4) \\
    &(e_1 \times o_2 , e_3 \times o_4) \\
    &(e_1 \times o_2 , o_3 \times o_4) \\
    &(o_1 \times o_2 , o_3 \times o_4)
\end{align}
In cases (2), (4), (5), the current player can win. \\
In cases (1), (3), (6), the current player cannot win. \\

\textbf{Proof:} \\

First, consider $(1 \times 1, 1 \times 1)$, this is the only way the game can end, when a player receives these rectangles, they cannot make a cut and loses. Since $1 = 2^0$ this is obviously case (1).\\

For case (1) that is not $(1 \times 1, 1 \times 1)$, there must be at least one side on one rectangle that can be cut. Let this be $e_1$ of $e_1 \times e_2$. Thus, $e_1 > 1$. This can cut the rectangle into either two, where both sides are 2-odd, 










Given rectangles $(1\times n, 1\times n)$, where $n \in E_2$. The current player cannot win. \\
Given rectangles $(1\times m, 1\times m)$, where $m \in O_2$. The current player can win. \\
Given rectangles $(1\times a, 1\times b)$, where $a,b \in \mathbb{N}$ and $a\neq b$. The current player can win. \\