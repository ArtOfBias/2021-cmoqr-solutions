\documentclass{article}
\usepackage[utf8]{inputenc}
\usepackage{fancyhdr}
\usepackage{lastpage}
\usepackage{amsmath}
\usepackage{amssymb}
\usepackage{graphicx}

\title{Repechage 2021}
\date{February 2021}

\usepackage{geometry}
\geometry{
 letterpaper,
 left=20mm,
 right=20mm,
 top=20mm,
 bottom=20mm
 }

\setlength{\parindent}{0pt}

\pagestyle{fancy}
\fancyhf{}

\chead{Question 4 \\}
\rfoot{{\thepage} of \pageref{LastPage}}

\begin{document}
\begin{figure}[htp]
    \centering
    \includegraphics[width=200pt]{Q4_img}
    \label{fig:diagram}
\end{figure}

Let $O_1$ denote the incircle of $\triangle ABC$ and $O_2$ denote the circumcircle of $\triangle ABC$.\\
Let $d$ denote the length of $\overline{IO}$, $r$ the radius of $O_1$, and $R$ the radius of $O_2$. \\

Since $P$ and $Q$ are on $O_1$, and $\overline{PQ}$ passes through its centre, $I$, $\overline{PQ}$ is a diameter of $O_1$. \\
Thus, $PI = QI = r$. Since $\overline{PQ} \perp \overline{IO}$, $\angle PIO = \angle QIO = 90^{\circ}$. \\
Therefore, $\triangle PIO \cong \triangle QIO$ by SAS congruence. Hence, $OP = OQ$. \\

Substituting $d$ and $r$ and using the Pythagorean Theorem in right triangle $\triangle PIO$, we obtain:
\begin{align*}
    OP^2 &= PI^2 + IO^2 \\
    OP &= \sqrt{r^2 + d^2}.
\end{align*}
By Euler's Theorem, we know that $d^2 = R^2 - 2Rr$. \\
Note that $R>r$, so $R-r>0$, thus:
$$OP = \sqrt{r^2 + R^2 - 2Rr} = \sqrt{(R - r)^2} = R - r.$$
Recall that $OP = OQ$. Thus, the perimeter of $\triangle OPQ$ is:
$$OP + PQ + OQ = (R-r) + (2r) + (R-r) = 2R,$$
which is equal to the diameter of the circumcircle.
\end{document}