\documentclass{article}
\usepackage[utf8]{inputenc}
\usepackage{fancyhdr}
\usepackage{lastpage}
\usepackage{amsmath}

\title{Repechage 2021}
\date{February 2021}

\usepackage{geometry}
\geometry{
 letterpaper,
 left=20mm,
 right=20mm,
 top=20mm,
 bottom=20mm
 }

\setlength{\parindent}{0pt}

\pagestyle{fancy}
\fancyhf{}

\chead{Question 6 \\}
\rfoot{{\thepage} of \pageref{LastPage}}

\begin{document}

Rearrange the original equation:
\begin{align*}
    w^2 + 11x^2 -8y^2 -12yz - 10z^2 = 0 \\
    w^2 + 11x^2 = 8y^2 + 12yz + 10z^2. \tag{i}
\end{align*}
Take modulo $4$:
\begin{align*}
    w^2 + 11x^2 &\equiv 8y^2 + 12yz + 10z^2 \pmod 4\\
    w^2 + 3x^2 &\equiv 2z^2 \pmod 4.
\end{align*}
$1$ and $0$ are the only quadratic residues modulo 4, thus: $w^2, x^2, z^2 \equiv 0,1 \pmod 4$ \\
Thus, the only solutions are:
\begin{align}
    w^2 &\equiv 1 \pmod 4,\ x^2 \equiv 1 \pmod 4,\ z^2 \equiv 0 \pmod 4 \\
    w^2 &\equiv 0 \pmod 4,\ x^2 \equiv 0 \pmod 4,\ z^2 \equiv 0 \pmod 4
\end{align}
\\
\underline{\textbf{In case (1)}, $w,x$ are odd and $y$ is even.} \\
Thus, let $w = 2i + 1$, $x = 2j + 1$, and $z = 2k$. Substitute this into (i) to obtain:
\begin{align*}
    (2i + 1)^2 + 11(2j + 1)^2 &= 8y^2 + 12y(2k) + 10(2k)^2 \\
    4i^2 + 4i + 44j^2 + 44j + 12 &= 8y^2 + 24yk + 40k^2 \\
    i(i + 1) + 11j(j+1) + 3 &= 2y^2 + 6yk + 10k^2.
\end{align*}
Take modulo 2 of the left side to obtain:
$$i(i + 1) + 11j(j+1) + 3 \equiv i(i + 1) + j(j+1) + 1 \pmod 2.$$
Since $i$ and $j$ are integers, $i(i + 1) \equiv j(j+1) \equiv 0 \pmod 2$. Thus:
$$i(i + 1) + j(j+1) + 1 \equiv 1 \pmod 2.$$
Modulo 2 of the right side produces:
$$2y^2 + 6yk + 10k^2 \equiv 0 \pmod 2.$$
This is a contradiction. Thus, case (1) is impossible. \\ \\

\underline{\textbf{For case (2)}, $w,x,z$ are all even.} Let $w = 2l,\ x = 2m,\ z = 2n$. Substitute this into (i) and simplify:
\begin{align*}
    (2l)^2 + 11(2m)^2 &= 8y^2 + 12y(2n) + 10(2n)^2 \\
    l^2 + 11m^2 &= 2y^2 + 6yn + 10n^2. \tag{ii}
\end{align*}
Now, take modulo $2$ of this equation:
\begin{align*}
    l^2 + 11m^2 &\equiv 2y^2 + 6yn + 10n^2 \pmod 2 \\
    l^2 + m^2 &\equiv 0 \pmod 2.
\end{align*}
The only solutions are:
\begin{align*}
    l^2 \equiv 1 \pmod 2 \ \text{and}\ m^2 \equiv 1 \pmod 2 \tag{2.1} \\
    l^2 \equiv 0 \pmod 2 \ \text{and}\ m^2 \equiv 0 \pmod 2. \tag{2.2}
\end{align*}
\newpage
\underline{\textbf{Consider case (2.1)}, $l,m$ must both be odd,} so let $l = 2p + 1$ and $m = 2q + 1$. Substitute this into (ii) and simplify:
\begin{align*}
    (2p + 1)^2 + 11(2q + 1)^2 &= 2y^2 + 6yn + 10n^2 \\
    2p^2 + 2p + 22q^2 + 22q + 6 &= y^2 + 3yn + 5n^2. \tag{iii}
\end{align*}
The left side is even, so both $y$ and $n$ must be even. Let $y = 2r$, $n = 2s$, substitute into (iii) and simplify:
\begin{align*}
    2p^2 + 2p + 22q^2 + 22q + 6 &= (2r)^2 + 3(2r)(2s) + 5(2s)^2 \\
    2p^2 + 2p + 22q^2 + 22q + 6 &= 4r^2 + 12rs + 20s^2
\end{align*}
Modulo 4 of the right side produces:
$$4r^2 + 12rs + 20s^2 \equiv 0 \pmod 4.$$
Take modulo 4 of the left side to obtain:
\begin{align*}
    2p^2 + 2p + 22q^2 + 22q + 6 &\equiv 2p^2 + 2p + 2q^2 + 2q + 2 \pmod 4 \\
    &\equiv 2p(p + 1) + 2q(q+1) + 2 \pmod 4
\end{align*}

Since $p$ and $q$ are integers, $2p(p + 1) \equiv 2q(q+1) \equiv 0 \pmod 4$. Thus:
$$2p^2 + 2p + 2q^2 + 2q + 2 \equiv 2 \pmod 4.$$
This is a contradiction. Thus, case (2.1) is impossible.\\ \\

\underline{\textbf{Consider case (2.2)} In equation (ii), both $l,m$ must be even.} \\
Use Fermat's Infinite Descent method to prove case (2.2) is impossible. \\

Rearrange equation (ii) to get:
$$l^2 + 11m^2 = y^2 + (y^2 + 6yn + 9n^2) + n^2 = y^2 + (y + 3n)^2 + n^2.$$

If $l$ = 0 and $m$ = 0, then $y =0$ and $n =0$. This produces $(w,x,y,z) = (0,0,0,0)$ which is excluded. Therefore, $l$ and $m$ cannot both be zero. Furthermore, the nonzero integer solutions for $l$ or $m$ (if exists) can be positive or negative, but appear in pairs with the same absolute value. Without loss of generality, consider the positive integer solutions here. \\

Assume there exists a solution $(l_1,m_1,y_1,n_1)$ for (ii), where $l_1$ is the smallest positive integer solution for $l$ when $l > 0$, or $m_1$ is the smallest positive integer solution for $m$ when $m > 0$. \\
By definition of case (2.2), both $l_1$, $m_1$ must be even, so let $l_1 = 2a$, $m_1 = 2b$. Substitute into (ii) and simplify:
\begin{align*}
    (2a)^2 + 11(2b)^2 &= 2y_1^2 + 6y_1n_1 + 10n_1^2 \\
    2a^2 + 22b^2 &= y_1^2 + 3y_1n_1 + 5n_1^2. \tag{iv}
\end{align*}
The left side is even, so $y_1$ and $n_1$ must both be even. Let $y_1 = 2c$, $n_1 = 2d$, substitute into (iv) and simplify:
\begin{align*}
    2a^2 + 22b^2 &= (2c)^2 + 3(2c)(2d) + 5(2d)^2 \\
    a^2 + 11b^2 &= 2c^2 + 6cd + 10d^2.
\end{align*}
Note that the above equation is the same as equation (ii). This means $(a,b,c,d)$ is also a solution to equation (ii), with $a = \frac{l_1}{2}$ and $b = \frac{n_1}{2}$. This contradicts with our assumption that $l_1$ is the smallest positive integer solution for $l$, or $m_1$ is the smallest positive integer solution for $m$. This proves that there are no positive integer solutions for l or m for case (2.2).\\
Similarly, it can be proved that there are no negative integer solutions for $l$ or $m$ for case (2.2).\\
Therefore, there are no integer solutions for $l$ or $m$ for case (2.2), which means there are no integer solutions for $(l,m,y,n)$ for case (2.2).\\

\textbf{In summary, case (1), case (2) [consisting of (2.1) and case (2.2)] are all impossible. Thus, there is no nonzero integer solution for the original equation.}

\end{document}

$w^2, x^2, z^2 \equiv 0,1 \pmod 4$ as $1$ and $0$ are the quadratic residues of $\pmod 4$.

Both sides are the sum of squares. If both $l$ and $m$ are zero, $y$ and $n$ must be zero. This produces $(w,x,y,z) = (0,0,0,0)$ which is excluded. Therefore, $l$ and $m$ cannot both be zero. \\
Consider the left side of (ii), the solutions for $l$ and $m$ can be positive or negative, but appear in pairs with the same absolute value. Without loss of generality, consider the positive integer solutions here. \\



Thus, if $(a_1,b_1,c_1,d_1)$ is an nonzero integer solution to (ii), then $\left( \frac{a_1}{2}, \frac{b_1}{2}, \frac{c_1}{2}, \frac{d_1}{2} \right)$ must also be such a solution. \\
Now suppose this nonzero integer solution $(a_1,b_1,c_1,d_1)$ exists, then an integer solution $\left( \frac{a_1}{2}, \frac{b_1}{2}, \frac{c_1}{2}, \frac{d_1}{2} \right)$ is a solution, and so is $\left( \frac{a_1}{4}, \frac{b_1}{4}, \frac{c_1}{4}, \frac{d_1}{4} \right)$, $\left( \frac{a_1}{8}, \frac{b_1}{8}, \frac{c_1}{8}, \frac{d_1}{8} \right)$, so on.
Consider $l$ in each solution except the first, that is, $\frac{a_1}{2}, \frac{a_1}{4}, \frac{a_1}{8} \dots$, this is an infinite geometric sequence of distinct integers, call this $S_a$.
Since dividing by $2$ does not change the sign of a number, each term of $S_a$ lies entirely between $a_1$ and $0$. However, since $a_1$ is a finite integer, this interval contains a finite number of distinct integers, while the sequence contains infinite many distinct integers. Thus, such a sequence cannot exist, and thus no solutions to (ii) exist.
